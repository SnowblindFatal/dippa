\chapter{Methods}
\label{chapter:methods}

The basic problem can be divided into two parts. First is abstracting the business data into numbers and the second is applying the found numbers into the chosen vehicle routing problem algorithm. The business data includes locations of the job targets and the depots and the man-hour and equipment requirements of the jobs.  

\section{Abstracting the problem field into numerical data}

Raw data from the client company's database is unusable by itself. It needs to be transformed into abstract numbers that can be used as parameters for the routing heuristic. The goal is to minimise the number of variables to prevent the actual algorithm from becoming overly complex while still ensuring that all available information is taken into account when generating the solution. 

\subsection{Generating the distance matrix}

One problem in the project is transforming the location data, mainly addresses, into form that can be more readily be used programmatically. The goal is to produce a matrix that lists the travelling time units between all the targets, including the depot. Because of the large number of targets, it is reasonable to optimise this step by grouping nearby targets together. Grouping should be done so that the traveling time between them is negligible and thus not much information is lost by the operation. The end result is that as far as the algorithm is concerned, the grouped targets occupy the same address.

Because the addresses of the customers are inputted by hand into the database of the client company, not all of them are valid. Some have typos, while others may have the wrong municipality due misinformation or the frequent merges in the country. To mitigate this problem, the user can re-enter the address in case the routing service does not recognise it. Another option is to group the job with already existing jobs.

Choosing a suitable routing service is an important decision to be made in the early development. Generating the many-to-many traveling time matrix required in this project quickly cause an increase in the number of queries required. The licence fees which are typically based on the number of queries made within a time period must be kept reasonable.

Naturally the routing service has to support Finland's addresses and roadmap. Another important requirement is that it can estimate the traveling time between targets, because distance by itself is not very descriptive due to varying speed limits on different roads. Traffic congestion will probably be impossible to take into account because the driving will take place weeks into the future and also because the exact time and date are not known at the time of the query. 

One possible way to take congestion into account would be to redo the queries after an initial solution has been generated if the routing service can predict traffic based on previous dates' data and then adjust the solution if needed. However, the potential benefits would be limited at best and would not justify the increased complexity.


\subsection{Calculating job costs}

NOTE: number of technicians might always be one. This needs checking. Overall this is a very crude sketch of the subsection. 

To prevent the algorithm from becoming overly complex, the time and resource requirements of a single job should be simplified as much as possible. The input:

\begin{itemize}
\item Types and number of tasks involved in the job
\item Installation items and equipment required.
\item Minimum number of technicians required.
\item The assigned technician's performance if a specific technician is required for the job.   
\end{itemize}

The end result should be the following parameters per job target:

\begin{itemize}
\item Vehicle capacity requirements of the installation items and other equipment.
\item Minimum number of technicians required at the job.
\item Total time requirement for the technicians assuming that the number of technicians is the minimum number.  
\end{itemize}

Each task type has a unique time cost associated with it which is then adjusted with various factors. Different house types, floors and other environmental conditions affect on how much time is required per task. The type and size of item to be installed into the house also plays a role. 

The number of technicians required at a job target is a hard parameter, in that one technician working twice as long will not be able to complete the task. If even a single task at the target requires more than one technician, the whole job will be considered a two-person job. 

Two people do not perform twice as fast as a single person and different tasks at a job benefit different amounts from an increased workforce. Therefore the benefit from additional technicians should be estimated per type of task. 



% TODO: generate a generic way to include new tasks, resources, house types and how they affect the total job requirements.
% TODO: how the algorithm parameters fit into this. 