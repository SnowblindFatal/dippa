\chapter{Introduction}
\label{chapter:intro}

The vehicle routing problem (VRP) is a topic that has been researched with great intensity for a long time. The basic idea behind vehicle routing problem is that given a depot and a set of target locations, the depot must dispatch a number of vehicles so that each location is visited once and that the vehicles finally return to the depot. The total traveling cost should be minimised. Vehicle routing problem is a classic problem first introduced by Dantzig and Ramser in 1959. \cite{dantzig1959truck} It is closely related to the even more well-known traveling salesman problem, first presented by Hassler Whitney in 1934 \cite{flood1956traveling}. 

Because VRP has numberous applications in many fields, there is much to benefit from the ability to create good solutions to it. Since VRP is an NP-hard problem as it is a superset of the traveling salesman problem, it is typically necessary to use heuristics to get a solution, as finding a perfect solution would be too intensive computationally \cite{laporte2007you}. The goal of this thesis is to survey the existing heuristics to the problem and use them in practice. I will analyse the quality of the solutions and computational cost of the algorithms.  

\section{Problem statement} 
In many businesses from postal services to food delivery and cargo transport a lot of time is spent traveling between the customers or depots. There is a good motivation to reduce the time spent traveling, because by itself it does not produce any value to the company, but rather generates only costs through wages, car usage and reserved equipment.  

While planning the routes by manual work is feasible for small businesses, as a company scales up and the number of customers and technicians increases, automation becomes more and more important. At some point the overhead of implementing an automated system for the planning overcomes the cost of the ever increasing planning work. 

Once the factors affecting the work are known, it is possible to abstract the company's operation to numbers which in turn can be used as input parameters for algorithms. Because the parameters affecting the route planning vary from business to business, not all algorithms will be feasible for all applications. However, there are numerous subcategories of the vehicle routing problem to address this issue.

\section{Structure of the Thesis}
Added after everything else.