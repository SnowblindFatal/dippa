\chapter{Evaluation}
\label{chapter:evaluation}


The main issues lie in usability. Currently the tool requires manual fetching of the input data in json format from the client company's system by adjusting the API address url to get the data from a specific date range. This data has to be then saved to a file and put to the program folder.

How my solution worked out. How comparable it is to existing models? What are the time requirements with the data given? How optimal are the solutions? Is the software future-proof?


\section{Customer's perspective}

The customer was able to run the program and make it produce the expected output. The customer was overall pleased with the program's output and found that it could be used for planning routes. 

This program, a preliminary demo of what is possible to do, caused interest the client company to further develop the work planning procedures used in the company and automate them further.

The biggest downside found by the client was that there can be quite a large gap in the earliest possible installation date between the jobs on a single route. This is because currently the program does not care about the earliest installation date except at results outputting. 



\section{Future development}

Probably the biggest room for development lies in usability. Using the program for a specific installer or automatically fetching the input data according to user preference would make the program's usage a lot more comfortable. The lack of user interface makes the program potentially confusing for a non-expert.

The program's error handling capabilities are also very limited. Most errors cause the program to crash or produce bad or nonexisting results without giving an understandable error message. This is not an issue for research use, but real world usage would definitely benefit from a more solid error handling.      

Currently the program uses very rough estimates for determining the time required at a worksite. 1 hour per item and a fixed 15 minute overhead work fine for testing purposes, but more detailed information would provide better and more optimised routes. This would be a rather simple task that would be achievable with the usage of configuration files, for example. 

The user could enter the average time requirements per installation item type based on past experience on worksites where the item has been installed. The effect of different house types and the floor number of the operation could be taken into account. Furthermore, the relative performance of technicians could be listed, allowing the scaling of the time requirement to get as accurate estimation as possible.

The client company's request for following the time windows more strictly is also a clear way to improve the results provided by the program. Considering that the jobs should start as soon as possible after the construction materials have been delivered to the worksite, the algorithm should try to group targets with near delivery dates onto same routes. Currently the algorithm can create routes such that the delivery of the installation materials of the earliest job is weeks before the delivery of the latest job, resulting in a scenario where the the materials will lie  unused at the worksite of the earlier job for weeks. This would lead to reduced customer satisfaction.

One way to address the aforementioned issue would be to assign a penalty to routes with a big variety in the earliest installation dates. This would result in the algorithm favouring routes where the jobs' earliest installation dates are grouped more tightly together. This would require implementing the aforementioned additional feature to the Jsprit library which might be a lot of work. 

Another way would be to group the installation jobs by their earliest installation dates. This would be easy to implement. However, it would be difficult to tell if a certain job would fit better in the next group, resulting in potentially unnecessarily unoptimal routes. This could be countered by running the algorithm multiple times and adjusting the date ranges of the groups between runs. Then the best routes could be picked from the results. However, this would require more complex results analysis due to jobs being found in multiple routes. 

The results output is currently just textual representation with the routes and the routes' job locations indented for clarity. A proper visual representation of the routes and suggested dates for them would make benefitting the results much easier. The Jsprit library supports exporting the results in xml format already, so producing a custom machine-readable output is not necessary, unless using a simpler schema or some format other than xml is required. 

The structure of the program would need refactoring, as it was developed for research purposes with a lot of trial and error. Though the codebase is small, less than 1000 lines, restructuring and otherwise cleaning the code is necessary if this software is to be used in a bigger context.


\section{Retrospective}

Overall, I feel like this project was a success. I reached the goals I had set beforehand and firmly believe that the future development of the program is not only possible, but would increase the potential 