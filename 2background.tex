\chapter{Vehicle routing problem introduction}
\label{chapter:background} 

\section{Problem statement}
Vehicle routing problem is a classic problem first introduced by Dantzig and Ramser in 1959\cite{dantzig1959truck}. It is closely related to the even more well-known traveling salesman problem, first presented by Hassler Whitney in 1934\cite{flood1956traveling}. 

Traveling salesman problem is a special case of vehicle routing problem, where the number of vehicles is 1, there is no upper limit on the distance of a route and the capacity of the vehicle is greater than the combined requirement of all the nodes in the network.



\section{Variations of the vehicle routing problem}
\subsection{Vehicle routing problem with multiple depots}
\subsection{Vehicle routing problem with pickup and delivery}
