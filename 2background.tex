\chapter{Vehicle routing problem introduction}
\label{chapter:background} 

\section{Classical vehicle routing problem}



The basic VRP can be defined with an graph $G = (V, A)$ where $V = \{v_1, v_2, v_3\dots\}$ represents the vertices, ie. the depot and targets for the visitations and $A = \{(v_i, v_j): i \neq j \}$ represents the arcs between the vertices. For every arc, there is a non-negative travel cost $C=(c_{ij})$. In this context, it is used to represent the traveling time between vertices. If the travel cost is symmetric between all pairs of vertices, an undirected graph $G = (V, E)$ can be used instead, where $E=\{(i, j) : i, j \in V, i < j\}$. \cite{laporte2007you}

The first vertex in $V$ represents the depot, where all the vehicles start from and where they must end their trips. $m$ identical vehicles have a capacity denoted by $Q$ which symbolises the resources available for each trip. In the classic VRP, $m$ is logically equivalent to the number of trips needed to visit every customer. With time windows this no longer holds true. Every customer vertex $i \in V\setminus\{0\}$ has a demand $q_i \leq Q$ associated with it. This symbolises the various resources, including time, required at the target. \cite{laporte2007you}

Vehicle routing problem is NP-hard as it is a superset of the traveling salesman problem. In traveling salesman problem, the number of vehicles $m = 1$, there is no upper limit on the costs of a route and the capacity of the vehicle is greater than the combined requirement of all the nodes in the network ($Q = \infty$). \cite{laporte2007you} 

Let $x_ij$ denote the number of trips made over the edge $[i, j]$ in a solution. The total cost that should be minimised is then:

\begin{equation}
\label{eq:baseformula1}
\displaystyle \sum_{[i,j] \in E} c_{ij}x_{ij}
\end{equation}

\noindent
The following conditions must hold true:

\begin{equation}
\label{eq:baseformula2}
\displaystyle \sum_{j \in E \setminus\{0\}} x_{0j} = 2m
\end{equation}

\noindent
Meaning that given $m$ trips, the sum of the trips made between the depot and other vertices is $2m$.


\begin{equation}
\begin{aligned}
\label{eq:baseformula3}
\displaystyle\sum_{i < k} x_{ik} + \displaystyle\sum_{j > k} x_{kj} = 2 && (k \in V \setminus\{0\})
\end{aligned}
\end{equation}

\noindent
Meaning that every vertex that is not the depot has two edges that are used in the solution. In other words, every non-depot vertex is visited exactly once.


\begin{equation}
\begin{aligned}
\label{eq:baseformula4}
\displaystyle\sum_{\substack{i \in S, j \notin S \\ 
\text{ or } i \notin S, j \in S}} x_{ij} \geq 2b(S) && (S \subset V \setminus\{0\})
\end{aligned}
\end{equation}

\noindent
$b(S)$ represents the minimum number of vehicles required to satisfy the needs of the customers of $S$. This constraint means the number of edges travelled between the vertices in the network contained in S and the vertices of the rest of the network has to be at least 2 times the minimum number of vehicles required to satisfy the needs of the customers of $S$. 



\begin{equation}
\begin{aligned}
\label{eq:baseformula5}
x_{i,j} = 0 \text{ or } 1 && (i, j \in V\setminus\{0\})
\end{aligned}
\end{equation}


\begin{equation}
\begin{aligned}
\label{eq:baseformula6}
x_{0,j} = 0, 1 \text{ or } 2 && (j \in V\setminus\{0\})
\end{aligned}
\end{equation}

\noindent
Meaning that every edge whose other end is the depot is travelled at most two times, while the rest of the edges are traveled once at most. \cite{laporte2007you}




 


\section{Variations of the vehicle routing problem}

The real life needs of route planning are rarely satisfied with the basic vehicle routing problem. There are typically additional constraints and an increased complexity that require expanding the problem statement. The vehicles used are probably not identical in terms of capacity and cost. The number of depots might also be greater than 1 and each target has to be allocated to one of them. \cite{salhi2014multi} In addition, there might be specific time windows during which the targets have to be visited \cite{ghoseiri2010multi}. All of these additional features can be combined into a multitude of variations of the VRP. As all the variations add to the complexity of the VRP, they too are NP-hard.

\subsection{Vehicle routing problem with time windows}

Time windows and other scheduling constraints are common in real world and thus including them in VRP is crucial in order to get results that are applicable in common business cases. These can include postal, food or cargo delivery and service visits where it is common that the customer wants the delivery to occur during a specific time. \cite{cordeau2000vrp}

To define vehicle routing problem with time windows (VRPTW), let $s_i$ denote the time that a vehicle has to spend at a target $i$ and $b_i$ denote the time at which the delivery of the service or goods begins. The earliest time the target accepts the start of the delivery is $e_i$ and the latest is $l_i$. If a vehicle arrives too early at $j$, it will have to wait so that the earliest time it can deliver the goods or services is $b_i = \max\{e_j, b_i + s_i + t_{ij}\}$. Here $t_{ij}$ represents the time required to travel between $i$ and $j$. \cite{solomon1987algorithms}

\subsection{Periodic vehicle routing problem}

While the classic VRP assumes a single time period, a single day for example, during which a deliveries have to be made, periodic vehicle routing problem (PVRP) defines a case with repeating time windows. Customers may need to be visited once or multiple times, and they may require the visit to occur on a specific weekday. \cite{blakeley2003optimizing} Common applications for the PVRP include waste collection, vending machine replenishment and cleaning service. PVRP is commonly combined with VRPTW as it is likely that in addition to wanting the visit to take place on a specific day, the customers also have requirements regarding the time of the visit. \cite{yu2011ant}

\subsection{Multi-depot Vehicle routing problem}

Multi-depot Vehicle routing problem (MDVRP) is a variation of VRP where there are additional depots which can be utilised as the start and end points of routes. A vehicle must start and end the route at the same depot. While at first it might seem that MDVRP can be just split into multiple single depot VRPs by assigning each target to the nearest depot, this leads to suboptimal solutions. \cite {salhi2014multi}


\subsection{Heterogenous fleet vehicle routing problem}

In real world scenarios, it is uncommon that all the vehicles in a fleet are identical. A typical variation of the VRP called heterogenous fleet vehicle routing problem (HVRP) one where the vehicles are not assumed to be identical. There is a fixed cost $f_t$ associated with each vehicle type $t$ and a variable cost per distance unit $g_t$. Vehicle types also have unique capacities, denoted by $Q_t$. \cite{gendreau1999tabu}

\subsection{Vehicle routing problem with pickup and delivery}

Vehicle routing problem with pickup and delivery (VRPPD) defines a VRP where a set of transportation requests must be satisfied. The requests all have a pickup point and a delivery point where the goods or passenger has to be brought. In case of passengers, a common application in real life can be found in taxi service. A courier service would be an example of a scenario where goods need to be transported from one point to another. Due to the nature of the applications for VRPPD, it is usually combined with VRPTW. The standard VRP is a VRPPD where the pickup point is always the depot and the delivery points are spread out elsewhere or vice versa. \cite{desaulniers2000vrp}








